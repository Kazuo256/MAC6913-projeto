%%% final.tex
%%%
%%% This LaTeX source document can be used as the basis for your technical
%%% paper or abstract. Intentionally stripped of annotation, the parameters
%%% and commands should be adjusted for your particular paper - title, 
%%% author, article DOI, etc.
%%% The accompanying ``template.annotated.tex'' provides copious annotation
%%% for the commands and parameters found in the source document. (The code
%%% is identical in ``template.tex'' and ``template.annotated.tex.'')

\documentclass[conference]{acmsiggraph}

\TOGonlineid{45678}
\TOGvolume{0}
\TOGnumber{0}
\TOGarticleDOI{1111111.2222222}
\TOGprojectURL{}
\TOGvideoURL{}
\TOGdataURL{}
\TOGcodeURL{}

\title{Rendering Metaballs with PBRT}

\author{Vinícius Vendramini\thanks{e-mail:todo@email.com}\\IME-USP \and Wilson K. Mizutani\thanks{e-mail:kazuo@ime.usp.br}\\IME-USP}
\pdfauthor{Vinícius Vendramini, Wilson K. Mizutani}

\keywords{implicit surface, metaball, mesh generation, ray tracing}

\begin{document}

%% \teaser{
%%   \includegraphics[height=1.5in]{images/sampleteaser}
%%   \caption{Spring Training 2009, Peoria, AZ.}
%% }

\maketitle

\begin{abstract}

Typically, three dimensional geometry is made through meshes, which requires
either modeling skills or data extraction from real bodies. This is specially
difficult for more organic or fluid structures, like creatures, liquids or
gases. Blinn \shortcite{Blinn:1982:GAS:965145.801290} proposes a way of defining
geometry though a particular type of implicit surface that makes easier the
representation of such geometries. In this work we strive for a thourough
renderization of these geometries using the awarded PBRT
\shortcite{Pharr:2010:PBR:1854996} render system. There are two methods covered:
one based on Blinn's original concept, and another using the marching cubes
algorithm \cite{Lorensen:1987:MCH:37402.37422}.

\end{abstract}

%\begin{CRcatlist}
%  \CRcat{I.3.3}{Computer Graphics}{Rendering Metaballs with PBRT}{Implicit surfaces}
%  \CRcat{I.3.7}{Computer Graphics}{Rendering Metaballs with PBRT}{Ray-tracing};
%\end{CRcatlist}

\keywordlist

%% Use this only if you're preparing a technical paper to be published in the 
%% ACM 'Transactions on Graphics' journal.

\TOGlinkslist

%% Required for all content. 

\copyrightspace

\section{Introduction}

\subsection{PBRT}

\subsection{Metaballs}

These representations
would be later called \textbf{metaballs}.

\subsection{Rendering methods}

\section{Marching cubes algorithm}

The marching cubes algorithm was introduced by Lorensen and Cline in 87 as an efficient way to render isosurfaces. The algorithm uses a divide-and-conquer approach by dividing the space that contains the surface into several much smaller cubes and then analysing the way the surface intersects with each cube. The nature of isosurfaces makes it easy to determine if each vertex is on one side or the other of any given cube. Based on the set of vertices that are on one side (ie the inside), the algorithm uses a preprocessed table containing several possible triangulations of the cubes to find the better triangular approximation to the piece of the isosurface contained within that cube. The way the triagulations are chosen guarantees the resulting surface will be topologically correct, even in the complicated ambiguous cases. 






\subsection{Ambiguous cases}

\begin{equation}
 \sum_{j=1}^{z} j = \frac{z(z+1)}{2}
\end{equation}

\begin{eqnarray}
x & \ll & y_{1} + \cdots + y_{n} \\
  & \leq & z
\end{eqnarray}

\section{Analytic intersection}


\begin{figure}[ht]
  \centering
  \includegraphics[width=1.5in]{images/samplefigure}
  \caption{Sample illustration.}
\end{figure}

\section{Implmentation}

\subsection{Marching cubes}

\subsection{Intersection}

\section{Conclusion}

Lorem ipsum dolor sit amet, consectetur adipisicing elit, sed do
eiusmod tempor incididunt ut labore et dolore magna aliqua. Ut enim ad
minim veniam, quis nostrud exercitation ullamco laboris nisi ut
aliquip ex ea commodo consequat. Duis aute irure dolor in
reprehenderit in voluptate velit esse cillum dolore eu fugiat nulla
pariatur. Excepteur sint occaecat cupidatat non proident, sunt in
culpa qui officia deserunt mollit anim id est laborum.

\subsection{Future works}

\section*{Acknowledgements}

%TODO

\bibliographystyle{acmsiggraph}
\bibliography{template}
\end{document}
